\documentclass[solid,math,chem,code,plot]{bmc}

\titlehead{Bespoke Multipurpose Class}
\title{BMC \hfill \fontsize{1.35cm}{1.35cm}\fontseries{t}\selectfont \emph{v}0.1}
\subtitle{Pretentiousness Given Form}
\author{tecosaur \footnotesize \newline Of Github}

\usepackage[symbol*]{footmisc}
\renewcommand{\thefootnote}{\fnsymbol{footnote}}
\usepackage[normalem]{ulem}

\begin{document}

\maketitle{}

\section*{Preamble}

\vspace{1cm}

Like with most things I didn't start out with the intent to end up this way.
Initially I had a slowly growing template that I used for most documents;
every so often I'd discover a package that did something I liked,
or a setting that I preferred to be non-default.
\emph{Every} time that happened I'd want to go through the current documents I was working on
and apply the latest revelations.
Then when revisiting old documents I'd want to get them `up to scratch'.
There would always be the odd document I forgot about, or line missed,
and so I quickly became tired of this process.

After realising that if I made a class and shoved it in my \verb|texmf|
directory that I'd be able to as many improvements as I like and they'd all
be applied when I recompiled, \emph{as well} as make initial configuration
greatly simplified --- I couldn't see a reason not to do it.

This class is very much written with my personal taste, and specific use case in mind.
While I try to keep things general it is very much built around my particular perspective.
As such it is reasonable to think that to the community as a whole the
self-importance in the name is a tad exaggerated or undeserved.
Considering that is also designed to not just convey information but also
designed to visually impress, the tagline ``Pretentiousness given form''
seems somewhat appropriate.

I'm pleased to say that I consider this a project a success (in those respects).
As I have largely drawn upon snippets of LaTeX floating around online
I though the least I could do is give others that same opportunity.
As such here you have an overview of my personal class
designed to work for all of the documents I produce.
In other words a \emph{bespoke, multipurpose class} --- or BMC for short.

\vspace{1cm}

Enjoy!

\vfill

tecosaur

\vspace{3cm}

\newpage
\fancytoc{}

\chapter{What This Does}

\section{Typography}

\subsection{Typefaces}

This package loads three typefaces.
\begin{enumerate}
    \item IBM Plex Serf
    \item IBM Plex Sans
    \item IBM Plex Mono
\end{enumerate}
I wanted a selection where serif, sans, and mono all mix well.
Ideally with a few weight variants.
Additionally I wanted a typographic style that meshed well with the
large class of documents I indented to use this for. IBM Plex seems like a good fit
(For more info see \autoref{sec:why-typefaces}). For all three of these a linespread of 1.15 is used.

\newcommand\setrow[1]{\gdef\rowmac{#1}#1\ignorespaces}
\newcommand\clearrow{\global\let\rowmac\relax}
\clearrow
\begin{table}[!htb]
    \centering
    \setlength{\tabcolsep}{4pt}
    \begin{tabular}{
        l
        >{\rowmac\ifbool{tabularTitleRow}{}{\fontseries{b}\selectfont}}l
        >{\rowmac\ifbool{tabularTitleRow}{}{\fontseries{sb}\selectfont}}l
        >{\rowmac\ifbool{tabularTitleRow}{}{\fontseries{mb}\selectfont}}l
        >{\rowmac\ifbool{tabularTitleRow}{}{\fontseries{tx}\selectfont}}l
        >{\rowmac}l
        >{\rowmac\ifbool{tabularTitleRow}{}{\fontseries{l}\selectfont}}l
        >{\rowmac\ifbool{tabularTitleRow}{}{\fontseries{el}\selectfont}}l
        >{\rowmac\ifbool{tabularTitleRow}{}{\fontseries{t}\selectfont}}l
        }
        \toprule
        Typeface & Bold & Semibold & Medium & Text & Regular & Light & Extra L & Thin \\
        \setrow{\ttfamily\scriptsize}\verb|\selectfont| & b & sb & mb & tx & m & l & el & t \\
        \midrule
        \setrow{\rmfamily}\!Plex Serif & Words & Words & Words & Words & Words & Words & Words & Words \\
        \setrow{\rmfamily\itshape}\!Plex Serif & Words & Words & Words & Words & Words & Words & Words & Words \\
        \arrayrulecolor{page}\midrule
        \setrow{\sffamily}Plex Sans & Words & Words & Words & Words & Words & Words & Words & Words \\
        \setrow{\sffamily\itshape}Plex Sans & Words & Words & Words & Words & Words & Words & Words & Words \\
        \midrule
        \setrow{\plexsanscondensed}Condensed & Words & Words & Words & Words & Words & Words & Words & Words \\
        \setrow{\plexsanscondensed\itshape}Condensed & Words & Words & Words & Words & Words & Words & Words & Words \\
        \midrule\arrayrulecolor{text}
        \setrow{\ttfamily}Plex Mono & Words & Words & Words & Words & Words & Words & Words & Words \\
        \setrow{\ttfamily\itshape}Plex Mono & Words & Words & Words & Words & Words & Words & Words & Words \\
        \bottomrule
    \end{tabular}
    % \caption{Font Styles}
    \label{table:font-styles}
\end{table}

\subsection{Roman Numerals}

While biblatex does provide handy roman numeral command,
it's nice to have them available regardless.
Hence this class provides them if they aren't already available.
To get upper case roman numerals use \mintinline{tex}{\RN{1978}}
to produce \RN{1978}, and \mintinline{tex}{\Rn{1978}}
to produce \Rn{1978}.

\begin{minted}[firstnumber=1037]{tex}
        \providecommand*{\RN}[1]{\expandafter\@slowromancap\romannumeral #1@}
        \providecommand*{\Rn}[1]{\romannumeral#1\relax}
\end{minted}

\subsection{Fake Small Caps}
Some fonts (such as IBM Plex) are not kind enough to provide small caps.
Simply using downscaled capitals is a barbaric and decidedly inferior solution.
So \mintinline{tex}{\fauxsc{}} is defined which, while not as nice as \emph{true}
small caps, is a darn sight better than just reducing the font size.
\vspace{-6pt}
\begin{center}
    \parbox{0cm}{
        \begin{tabbing}
            Barbaric Solution:\quad \= S{\footnotesize MALL} C{\footnotesize APS} \\
            \texttt{\textbackslash fauxsc{}}: \> \fauxsc{Small Caps}
        \end{tabbing}}
\end{center}

\subsection{Penalties}
The class sets new penalties.
\begin{minted}[firstnumber=482]{tex}
    \@beginparpenalty=10000 % don't like it when a paragraph title is on a different page to the start of the content
    \hyphenpenalty=500 % not a huge fan of hyphens, but they are worthwhile
    \righthyphenmin=4 % min letters post-hyphen
    \lefthyphenmin=4 % min letters pre-hyphen
\end{minted}

\subsection{Captions}
Caption labels are made to be upright sans-serif in the `text' style,
while captions are italic in the style of the body.
When captions flow beyond a single line, ragged right alignment is used.

\begin{minted}[firstnumber=520]{tex}
    \setkomafont{caption}{\itshape\color{text}}
    \setkomafont{captionlabel}{\fontfamily{\headingsFont}\fontseries{tx}\selectfont\upshape\color{text}}
    \captionsetup{justification=raggedright,singlelinecheck=true}
\end{minted}

\section{Colour}
\subsection{Theme Colours}
\newcommand{\colorBox}[2][page]{\begin{tikzpicture}
	\fill [#2] (0, 0) rectangle (2, 1);
	\node[text width=2] at (0.2,0.2) {\tiny\color{#1} \textsf{#2}};
\end{tikzpicture}}

This class makes use of the following defined colours.

\begin{center}
    \colorBox{primary}\colorBox{secondary}\colorBox{tertiary}\colorBox{quatrinary}
\quad
\colorBox{alternativePrimary}\colorBox{primaryVariant}\colorBox[text]{contrastColour}
\end{center}

Modifying these colours in the preamble affects the entire document.

\subsection{Colour Palette}
While xcolor and latex do already come with some `nice' shades,
nice colour themes may be found at \url{https://flatuicolors.com}.
These shades do not use the pretentious names listed, we just call them what they are
(e.g.\ nephritis \(*\to \) green). Instead of overwriting the pre-existing colour, these colours
have been differentiated by capitalisation, i.e.\ ``Green'' instead of `green'.

\subsection{Functional Colours}
This package has a few special colours that describe a particular aspect of a document,
such as \verb|href| and \verb|inlinemath|. For more information see \autoref{subsec:config-colours}.

\begin{multicols}{2}
    \centering
    \colorBox{Grey} \qquad \colorBox{LightGrey} \colorBox{DarkGrey} \\[-0.7mm]
    \colorBox{Red} \qquad \colorBox{LightRed} \colorBox{DarkRed} \\[-0.7mm]
    \colorBox{Yellow} \qquad \colorBox{LightYellow} \colorBox{DarkYellow} \\[-0.7mm]
    \colorBox{Blue} \qquad \colorBox{LightBlue} \colorBox{DarkBlue} \\[-0.7mm]
    \colorBox{Green} \qquad \colorBox{LightGreen} \colorBox{DarkGreen} \\[-0.7mm]
    \colorBox{Orange} \qquad \colorBox{LightOrange} \colorBox{DarkOrange} \\[-0.7mm]
    \colorBox{Purple} \qquad \colorBox{LightPurple} \colorBox{DarkPurple} \\[-0.7mm]
    \colorBox{Cyan} \qquad \colorBox{LightCyan} \colorBox{DarkCyan}
\end{multicols}

\section{Mathematics}

This class makes a few additions, and one or two modifications to Mathematics.

\subsection{Modifications}

\paragraph{Less/greater than or equal}
The less than or equal, and greater than or equal symbols are changed as such:
\vspace{-6pt}
\begin{center}
    \parbox{0cm}{
    \begin{tabbing}
        \(*\leq \) \quad \= new\quad \= \(*\geq \) \kill
        \(*\oldleq \) \> old \> \(*\oldgeq \) \\
        \(*\leq \) \> new \> \(*\geq \)
    \end{tabbing}}
\end{center}

\paragraph{Inline math colour}
After interspersing maths and text a fair bit I've begun to think there's some
merit to the Beamer `make all maths a different colour' approach.
So I've redefined the LaTeX inline math command such that
\mintinline{tex}{\(a^x + bx + c\)} now becomes \(ax^2 + bx + c\).
Avoiding this is easy, just change the colour of \mintinline{tex}{inlinemath}
in the preamble like so \mintinline{tex}{\colorlet{inlinemath}{text}}
 and you won't notice this exists.
For once-offs I've defined a stared variant \mintinline{tex}{\(* a^x + bx + c\)}
which produces the normal \(*ax^2 + bx + c\).

\begin{minted}[firstnumber=1030]{tex}
    \renewrobustcmd{\(}{\@ifstar\@inlinemath\@@inlinemath}
    \DeclareRobustCommand{\@inlinemath}{\relax\ifmmode\@badmath\else$\fi}
    \DeclareRobustCommand{\@@inlinemath}{\relax\ifmmode\@badmath\else$\fi\color{inlinemath}}
\end{minted}

\paragraph{Matrix environment}
The default for matrices (using \mintinline{tex}{\begin{bmatrix}} or similar)
is left aligned values, with no option to change this.
This class adds an optional parameter to change the alignment,
(\mintinline{tex}{\begin{bmatrix}[r]}), and defaults to right aligned.

\begin{multicols}{2}
    \subparagraph{Old}
    \[
        \begin{bmatrix}[l]
            3 & -2 \\
            -1 & 7 \\
        \end{bmatrix}
    \]

    \columnbreak

    \subparagraph{New}
    \[
        \begin{bmatrix}
            3 & -2 \\
            -1 & 7 \\
        \end{bmatrix}
    \]
\end{multicols}

\subsection{Additions}

\paragraph{\rlap{Deliminators}\hspace*{8em}}
\begin{tabular}{p{4.2em}>{\(*\to\quad \)}p{4em}}
    \mintinline{tex}{\abs{x}} & \(*\abs{x} \) \\
    \mintinline{tex}{\norm{x}} & \(*\norm{x}\)
\end{tabular}

\footnotetext[2]{Can also be used outside of math mode.\label{fn:1}}
\paragraph{\rlap{Sets\footref{fn:1}}\hspace*{8em}}
\begin{tabular}{p{4.2em}>{\(*\to\quad \)}p{4em}p{4em}>{\(*\to\quad \)}p{4em}} % chktex 21
    \mintinline{tex}{\RR} & \RR & \mintinline{tex}{\RR[n]} & \RR[n] \\
    \mintinline{tex}{\NN} & \NN & \mintinline{tex}{\NN[n]} & \NN[n] \\
    \mintinline{tex}{\ZZ} & \ZZ & \mintinline{tex}{\ZZ[n]} & \ZZ[n] \\
    \mintinline{tex}{\QQ} & \QQ & \mintinline{tex}{\QQ[n]} & \QQ[n] \\
    \mintinline{tex}{\CC}& \CC & \mintinline{tex}{\CC[n]} & \CC[n] \\
\end{tabular}

\paragraph{\rlap{Differential d}\hspace*{8em}}
\begin{tabular}{p{4.2em}>{\(*\to\quad \)}p{4em}}
\mintinline{tex}{\dd}\footref{fn:1} & \(*\dd \) % chktex 21
\end{tabular}

\paragraph{\rlap{Stats Operators}\hspace*{8em}}
\begin{tabular}{p{4.2em}>{\(*\to\quad \)}p{4em}}
    \mintinline{tex}{\Var} & \(*\Var \) \\
    \mintinline{tex}{\Cov} & \(*\Cov \) \\
    \mintinline{tex}{\E} & \(*\E \)
\end{tabular}

\paragraph{\rlap{Others}\hspace*{8em}}
\begin{tabular}{p{4.2em}>{\(*\to\quad \)}p{4em}}
    \mintinline{tex}{\qed}\footref{fn:1} & \(*\qed \) \\
    \mintinline{tex}{\qedhere}\footref{fn:1} & \(*\qedhere \) \\
    \mintinline{tex}{\Lap} & \(*\Lap \)
\end{tabular}

\section{Code}

This package spends a few lines tweaking
the minted and tcolorbox config to get code
blocks to look rather nice.

For example:
\begin{minted}[escapeinside=||,highlightlines={8,17}]{tex}
    \section{Code}

    This package spends a few lines tweaking
    the minted and tcolorbox config to get code
    blocks to look rather nice.

    For example:
    \begin{minted}[escapeinside=||,highlightlines={8,17}]{tex}
        \section{Code}

        This package spends a few lines tweaking
        the minted and tcolorbox config to get code
        blocks to look rather nice.

        For example:
        |\dots|
    \end||{minted}
\end{minted}

\section{Chemistry}
When the \mintinline{tex}{chem} option is used, \mintinline{tex}{mhchem}
is loaded with the configuration, however \mintinline{tex}{chemfig}
undergos a few modifications to make the results look cleaner.

\chemfig{*6(-=-(-COOH)=-=)}
\chemfig{X-[1]=^[-1]-[1]-[-1]-[1]-[-1]=^[1]-[-1]}
\qquad
\chemfig{-[1](=[3]O)-[-1]}

\begin{minted}[firstnumber=310]{tex}
	\setchemfig{
		chemfig style={line width=0.06642 em},  % 'Line Width'
		angle increment=30,
		double bond sep=0.35700 em,  % 'Bond Spacing'
		atom sep=1.78500 em,  % 'Fixed Length'
		bond offset=0.18265 em  % 'Margin Width'
	}
	\renewcommand*\printatom[1]{\small\ensuremath{\mathsf{#1}}}
\end{minted}

\section{Links and Metadata}
\renewcommand{\ULthickness}{1pt}
\def\soutt{\bgroup \ULdepth=-.45ex \ULset}
Both the \mintinline{tex}{hyperref} and \mintinline{tex}{hyperxmp} packages are used.
The widely used hyperref package of course provides hyperlinks.
This is \soutt{abused} used to add a few extra links; specificly
every page number is a link to the TOC, and the text of every header
links to the relevant chapter page.
This allows you to jump all over the document in just a few clicks.

The hyperxmp package is rather handy for setting a few fields of pdf metadata.
Using the \mintinline{tex}{\title}, \mintinline{tex}{\author}, and \mintinline{tex}{\subtitle}
attributes it sets the relevant metadata fields.
\enlargethispage{\baselineskip}

\chapter{Boring Info}

\section{Class Options}

This class builds off \mintinline{tex}{scrartcl},
any other options than those listed here will just be passed through.

\subsection{Main Styling}

\paragraph{\ttfamily dark}
Switches to a dark version of the style
\paragraph{\ttfamily solid}
Uses style with solid title page, and wide stripes on chapter pages, with solid colour bar at top of pages
\paragraph{\ttfamily stripe}
Uses plain background on title page, and thin stripes on chapter pages
\paragraph{\ttfamily article}
Use \mintinline{tex}{scrartcl} class instead of \mintinline{tex}{scrrept}
\paragraph{\ttfamily notes}
Move the margins to make room for notes

\subsection{Fonts}
\subsubsection{Body Text}

\paragraph{\ttfamily serif}
{\rmfamily Use serif font as main}
\paragraph{\ttfamily sans}
{\sffamily Use sans font as main}
\paragraph{\ttfamily mono}
{\ttfamily Use mono font as main}

\subsubsection{Math}

\paragraph{\ttfamily math-serif}
Same as with math option but forcing a serif font
\paragraph{\ttfamily math-sans}
Same as with math option but forcing a sans font
\paragraph{\ttfamily math-mono}
Same as with math option but forcing a mono font

\subsection{Headings}

These options set the style of the following components
\begin{itemize}
    \item \mintinline{tex}{\chapter} through to \mintinline{tex}{\subparagraph}
    \item The page head, and page number
    \item Caption labels
\end{itemize}

\paragraph{\ttfamily headings-serif}
Not a default.
\paragraph{\ttfamily headings-sans}
Default when either \mintinline{tex}{serif}
or \mintinline{tex}{sans} options are given.
\paragraph{\ttfamily headings-mono}
Default when either \mintinline{tex}{mono} option is given.

\subsection{Package Related}

\paragraph{\ttfamily chem}
Load and configure \mintinline{tex}{mhchem} and \mintinline{tex}{chemfig} packages

\paragraph{\ttfamily code}
Load and configure minted package

\paragraph{\ttfamily plot}
Load and configure \mintinline{tex}{pgfplots} package

\paragraph{\ttfamily math}
Load and configure some mathematical packages, and set font to match main text font (also see \mintinline{tex}{math-serif} etc.)

\newpage
\section{Packages Used}

\subsection{Overview}

\begin{center}
    \small
    \setlength{\tabcolsep}{0pt}
    \begin{tabular}{>{\hspace{3pt}\normalsize}l>{\hspace{5pt}}*{5}{p{7.9em}}}
        \toprule
        Category & \multicolumn{5}{l}{\normalsize Packages} \\
        \midrule
        General & \hyperref[par:etoolbox]{etoolbox} & \hyperref[par:xpatch]{xpatch} & \hyperref[par:Silence]{Silence} & \hyperref[par:ifdraft]{ifdraft} & \hyperref[par:geometry]{geometry}  \\
        & \hyperref[par:titlesec]{titlesec} & \hyperref[par:titletoc]{titletoc} &\hyperref[par:framed]{framed} & \hyperref[par:textpos]{textpos} & \hyperref[par:calc]{calc} \\
         & \hyperref[par:xcolor]{xcolor} & \hyperref[par:tikz]{tikz} & \hyperref[par:hyperref]{hyperref} & \hyperref[par:hyperxmp]{hyperxmp} & \hyperref[par:scrlayer-scrpage]{scrlayer-scrpage} \\
         \arrayrulecolor{page}\midrule
        Text & \hyperref[par:microtype]{microtype} & \hyperref[par:setspace]{setspace} & \hyperref[par:plex-serif]{plex-serif} & \hyperref[par:plex-sans]{plex-sans} & \hyperref[par:plex-mono]{plex-mono} \\
        & \hyperref[par:multicol]{multicol} & \\
         \midrule
         Table & \hyperref[par:booktabs]{booktabs} & \hyperref[par:tabularx]{tabularx} & \hyperref[par:longtable]{longtable}  \\
         \midrule
         Graphics & \hyperref[par:graphicx]{graphicx} & \hyperref[par:grffile]{grffile} & \hyperref[par:subcaption]{subcaption} & \hyperref[par:caption]{caption} \\
         \midrule
         infoBulle & \hyperref[par:infoBulle]{infoBulle} & \hyperref[par:marginInfoBulle]{marginInfoBulle} & \hyperref[par:fontawesome5]{fontawesome5}\\
         \midrule
         Chemistry & \hyperref[par:mhchem]{mhchem} & \hyperref[par:chemfig]{chemfig} \\
         \midrule
         Code & \hyperref[par:minted]{minted} & \hyperref[par:tcolorbox]{tcolorbox} \\
         \midrule\arrayrulecolor{text}
         Math & \hyperref[par:amssymb]{amsmath} & \hyperref[par:amssymb]{amssymb} & \hyperref[par:mathdesign]{mathdesign} & \hyperref[par:xfrac]{xfrac} & \hyperref[par:cancel]{cancel} \\
         & \hyperref[par:mathtools]{mathtools} & \hyperref[par:mathastext]{mathastext} & \hyperref[par:pgfplots]{pgfplots} \\
         \bottomrule
    \end{tabular}
\end{center}

\subsection{General Packages}

\paragraph{\ttfamily etoolbox}\label{par:etoolbox}
Provides LaTeX frontends to some of the new primitives provided by e-TeX
as well as some rather useful some generic tools --- namely,
\begin{itemize}
    \item Robust definitions
    \item Command Patching
    \item Command Protection
    \item Arithmetic counters and lengths
    \item Document Hooks
    \item Environment Hooks
\end{itemize}
\paragraph{\ttfamily xpatch}\label{par:xpatch}
Extends the command patching provided by \hyperref[par:etoolbox]{etoolbox}
\paragraph{\ttfamily Silence}\label{par:Silence}
Allows me to ignore expected warnings.
\paragraph{\ttfamily ifdraft}\label{par:ifdraft}
To make it easy to change things up a bit more than usual for draft mode.
\paragraph{\ttfamily scrlayer-scrpage}\label{par:scrlayer-scrpage}
To allow for those lovely headers and footers.
\paragraph{\ttfamily geometry}\label{par:geometry}
Loaded with options,
\begin{minted}[firstnumber=507]{tex}
    a4paper, ignoreheadfoot, left=\leftmargin, right=\rightmargin, top=2cm, bottom=3.5cm, headsep=1cm
\end{minted}
\paragraph{\ttfamily titlesec}\label{par:titlesec}
Allows for customisation of \mintinline{tex}{\chapter} etc.
Was originally used for all section commands, but now all except for
\mintinline{tex}{\chapter} have been transitioned to KOMA-script.
\paragraph{\ttfamily titletoc}\label{par:titletoc}
Allows significant tweaking to how the table of contents looks.
\paragraph{\ttfamily framed}\label{par:framed}
Facilitate the definition of new environments that take multi-line material, wrap it with some
non-breakable formatting (some kind of box or decoration) and allow page breaks in the material
\paragraph{\ttfamily textpos}\label{par:textpos}
Facilitates placement of boxes at absolute positions on the LaTeX page.
Loaded with options \mintinline{tex}{absolute,overlay}
\paragraph{\ttfamily hyperref}\label{par:hyperref}
Used to produce all sorts of hyperlinks in a document.
Loaded with option \mintinline{tex}{pdfa}
\paragraph{\ttfamily hyperxmp}\label{par:hyperxmp}
Improves metadata setting with hyperref.
\paragraph{\ttfamily calc}\label{par:calc}
Adds infix expressions to perform arithmetic on the arguments of the LaTeX commands
\mintinline{tex}{\setcounter}, \mintinline{tex}{\addtocounter}, \mintinline{tex}{\setlength}, and \mintinline{tex}{\addtolength}
\paragraph{\ttfamily xcolor}\label{par:xcolor}
Provides all sorts of colour use and mixing capabilities.
\paragraph{\ttfamily tikz}\label{par:tikz}
It's tikz. You can't draw anything without it.

\subsection{Text}

\paragraph{\ttfamily microtype}\label{par:microtype}
Always good to have. It simply makes text look better, specificity it applies the following,
\begin{itemize}
    \item Character protrusion
    \item Font expansion
    \item Adjustment of interword spacing and kerning
    \item Letterspacing
\end{itemize}
Configured with,
\begin{minted}[firstnumber=215]{tex}
activate={true,nocompatibility},final,tracking=true,kerning=true,spacing=true
\end{minted}
\paragraph{\ttfamily plex-serif}\label{par:plex-serif}
\paragraph{\ttfamily plex-sans}\label{par:plex-sans}
\paragraph{\ttfamily plex-mono}\label{par:plex-mono}
\paragraph{\ttfamily setspace}\label{par:setspace}
Provides an easy way to set line spacing with commands such as
\mintinline{tex}{\doublespacing} and \mintinline{tex}{\setstretch{1.25}}.
\paragraph{\ttfamily multicol}\label{par:multicol}
Split text into multiple columns (up to 10).

\subsection{Table-related}

\paragraph{\ttfamily booktabs}\label{par:booktabs}
Contribues different width \mintinline{tex}{\hline} variants.
\paragraph{\ttfamily tabularx}\label{par:tabularx}
Adds the \mintinline{tex}{tabularx} environment which has its width explicitly set,
\mintinline{tex}{X} column type which automatically determines its width based on its contents.
\paragraph{\ttfamily longtable}\label{par:longtable}
Provides a good way of allowing tables to spread over multiple pages.

\subsection{Graphics and Figures}

\paragraph{\ttfamily graphicx}\label{par:graphicx}
Makes loading images (\mintinline{tex}{includegraphics}) work well.
\paragraph{\ttfamily grffile}\label{par:grffile}
This fixes the fix allowed filenames of graphicx.
\paragraph{\ttfamily caption}\label{par:caption}
Povides many ways to customise the captions in floating environments like figure and table,
and cooperates with many other packages.
Facilities include rotating captions, sideways captions, continued captions (for tables or figures that come in several parts).
Loaded with option \mintinline{tex}{hypcap=true}
\paragraph{\ttfamily subcaption}\label{par:subcaption}
Allows for typeseting of sub-figures and sub-tables.

\subsection{infoBulle}

\paragraph{\ttfamily fontawesome5}\label{par:fontawesome5}
Fontawesome 5, need I say any more?
\paragraph{\ttfamily infoBulle}\label{par:infoBulle}
\paragraph{\ttfamily marginInfoBulle}\label{par:marginInfoBulle}

\subsection{Chemistry}

\paragraph{\ttfamily mhchem}\label{par:mhchem}
Useful for simple inline chemistry.
\paragraph{\ttfamily chemfig}\label{par:chemfig}
Useful for chemical diagrams.

\subsection{Code}

\paragraph{\ttfamily minted}\label{par:minted}
Configured as follows,
\begin{minted}[firstnumber=534]{tex}
    \setminted{
        frame=none,
        % framesep=2mm,
        baselinestretch=1.2,
        fontsize=\footnotesize,
        highlightcolor=page!95!text!80!primary,
        linenos,
        breakanywhere=true,
        breakautoindent=true,
        breaklines=true,
        tabsize=4,
        xleftmargin=3em,
        autogobble=true,
        obeytabs=true,
        python3=true,
        texcomments=true,
        framesep=2mm,
        breakbefore=\\\.+,
        breakafter=\,
    }
\end{minted}

\paragraph{\ttfamily tcolorbox}\label{par:tcolorbox}
Used for prettifying the \mintinline{tex}{minted} environment.
Loaded with option \mintinline{tex}{many}

\subsection{Math Related}
These packages are loaded by the \mintinline{tex}{math}
option (or one of its derivatives).

\paragraph{\ttfamily amsmath,amssymb}\label{par:amssymb}
Extends the math commands and symbols in latex.
\paragraph{\ttfamily mathdesign}\label{par:mathdesign}
To use the Utopia font for math symbols.
\paragraph{\ttfamily xfrac}\label{par:xfrac}
Allows split level fractions \(*\sfrac{a}{b}\) better than \mintinline{tex}{{}^{a}/{}_{b}} can produce.
\paragraph{\ttfamily cancel}\label{par:cancel}
Allows for easy canceling within math like so ---
\(*\cancel{\eta}\) and \(*\cancelto{0}{\eta}\).
Loaded with option \mintinline{tex}{makeroom}
\paragraph{\ttfamily mathtools}\label{par:mathtools}
Provides a varienty of enhancements to make math \emph{even} better.
\begin{itemize}
    \item Extensible symbols, such as brackets, arrows, harpoons, etc.;
    \item Various symbols such as \mintinline{tex}{\coloneqq} (\(*\coloneqq\));
    \item Easy creation of new tag forms;
    \item Showing equation numbers only for referenced equations;
    \item Extensible arrows, harpoons and hookarrows;
    \item Starred versions of the amsmath matrix environments for specifying the column alignment;
    \item More building blocks: multlined, cases-like environments, new gathered environments;
    \item Maths versions of \mintinline{tex}{\makebox}, \mintinline{tex}{\llap}, \mintinline{tex}{\rlap} etc.;
    \item Cramped math styles; and more\dots
\end{itemize}
\paragraph{\ttfamily mathastext}\label{par:mathastext}
Uses relevant plex font for maths letters.
Uses options \mintinline{tex}{basic,italic,symbolgreek}.

\paragraph{\ttfamily pgfplots}\label{par:pgfplots}
Loaded by the \mintinline{tex}{plot} option.

\section{Configuration}

\subsection{Colours}
\label{subsec:config-colours}
\clearrow
\begin{longtable}{l>{\rowmac}p{10em}>{\rowmac}p{10em}}
    \toprule
    Name & Default (Light) & Default (Dark) \\
    \midrule
    \endfirsthead
    \toprule
    \setrow{\scriptsize} Name & Default (Light) & Default (Dark) \clearrow \\
    \midrule
    \setrow{\scriptsize} \hspace{1em}\vdots & \hspace{1em}\vdots & \hspace{1em}\vdots \clearrow \\
    \endhead
    \setrow{\scriptsize} \hspace{1em}\vdots & \hspace{1em}\vdots & \hspace{1em}\vdots \clearrow \\
    % \midrule
    \bottomrule
    \endfoot
    \bottomrule
    \endlastfoot
    \mintinline{tex}{text} & \#000000 & \#FCFCFC \\
    \mintinline{tex}{page} & \#FFFFFF & \#222222 \\
    \mintinline{tex}{href} & tertiary & secondary \\
    \mintinline{tex}{primaryVariant} & primary\linebreak[0]!75!Cream\linebreak[0]>twheel,-3,360 & \ditto \\
    \mintinline{tex}{inlinemath} & secondary\linebreak[0]!50!text & tertiary\linebreak[0]!50!text \\
    \rowcolor{tableheadcolor}
    \multicolumn{2}{l}{\fontseries{l}\selectfont infoBulle} & \\
    \mintinline{tex}{infoBulleBackground} & page!90!text & \ditto \\
    \mintinline{tex}{infoBulleText} & text & \ditto \\
    \mintinline{tex}{marginInfoBulleBackground} & page & \ditto \\
    \mintinline{tex}{marginInfoBulleText} & text & \ditto \\
    \mintinline{tex}{criticalColor} & Red & \ditto \\
    \mintinline{tex}{questionColor} & Purple & \ditto \\
    \mintinline{tex}{informationColor} & Green & \ditto \\
    \mintinline{tex}{checkColor} & Blue & \ditto \\
    \mintinline{tex}{warningColor} & Orange & \ditto \\
    \mintinline{tex}{tipsColor} & Purple & \ditto \\
    \mintinline{tex}{exampleColor} & Blue & \ditto \\
    \mintinline{tex}{mathematicalColor} & Orange & \ditto \\
    \mintinline{tex}{codeColor} & Grey & \ditto \\
\end{longtable}

\chapter{The Whys}

\section{Layout}
\section{Typefaces}
\label{sec:why-typefaces}
\section{Colour}

\end{document}

